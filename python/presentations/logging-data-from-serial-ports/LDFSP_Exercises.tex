\documentclass[a4paper,11pt]{article}
\usepackage[top=30pt,bottom=30pt,left=48pt,right=46pt]{geometry}
\usepackage{exsheets}
\usepackage{exsheets-listings}
\usepackage{minted}
\definecolor{lightyellow}{HTML}{FFFFE0}
\usepackage{fontspec}
\setsansfont{Open Sans}
\setmonofont{Ubuntu Mono}
\setmainfont{Liberation Serif}
\usepackage{mathtools}
\usepackage{unicode-math}
\setmathfont{Asana Math}
%NCAS Green colour
\definecolor{NCASgreen}{HTML}{186F4D}

\begin{document}
\section{Logging Data From Serial Ports Exercise}

{\center~Make sure you are using Python 2.7}

\begin{question}
Import the \texttt{serial} module and 
open the serial port with the appropriate parameters.
\end{question}
\newsavebox\serbasic
\begin{lrbox}{\serbasic}
\begin{minipage}{\textwidth}
\begin{minted}[bgcolor=lightyellow]{python}
#!/usr/bin/python2.7
import serial

ser = serial.Serial(
   port='/dev/ttyUSB0',
   baudrate=9600,
   bytesize=serial.EIGHTBITS,
   parity=serial.PARITY_NONE,
   stopbits=serial.STOPBITS_ONE

\end{minted}
\end{minipage}
\end{lrbox}


\begin{solution}
\noindent\usebox\serbasic
\end{solution}
\begin{question}
Get a reading from the temperature probe.
\end{question}
\begin{solution}
\texttt{print ser.read(size=8)}
\par
"8" here is specific to the Papouch thermometer device.
\end{solution}
\begin{question}
Add a date and time reading to your output, using sensible choices for 
format, timezone, etc.
\end{question}
\begin{question}
Prove to yourself that there is a potential problem with a "gap" between 
the reading and the timestamp.
\end{question}
\begin{solution}
Compare:

\texttt{print datetime.utcnow().isoformat(), ser.read(size=8)}


\noindent~and:


\texttt{datastring = ser.read(size=8)}\par
\texttt{print datetime.utcnow().isoformat(), datastring}

\noindent~(run each one a few times in succession, and look at the differences between
the timestamps)
\end{solution}
\begin{question}
Add a loop to your code to continuously log the reading and time. What would be a good exit condition? Hint: try \texttt{dir(serial.Serial)} to see what methods 
might be of use.
\end{question}
\newsavebox\serloop
\begin{lrbox}{\serloop}
\begin{minipage}{\textwidth}
\begin{minted}[bgcolor=lightyellow]{python}
while ser.isOpen():
   datastring = ser.read(size=8)
   print datetime.utcnow().isoformat(), datastring
\end{minted}
\end{minipage}
\end{lrbox}

\begin{solution}
Several ways, but the simplest is:

\noindent\usebox\serloop

\end{solution}

\begin{question}
Rewrite your code to use \texttt{readline()}.
\end{question}
\newsavebox\readlinebox
\begin{lrbox}{\readlinebox}
\begin{minipage}{\textwidth}
\begin{minted}[bgcolor=lightyellow]{python}
import io
…
sio = io.TextIOWrapper(io.BufferedRWPair(ser, ser, 1), encoding='ascii', newline='\r')

while ser.isOpen():
   datastring = sio.readline()
   print datetime.utcnow().isoformat(), datastring
\end{minted}
\end{minipage}
\end{lrbox}

\begin{solution}
\noindent\usebox\readlinebox
\end{solution}
\begin{question}
Alter your code to write the data out to a file.
\end{question}

\newsavebox\logfile
\begin{lrbox}{\logfile}
\begin{minipage}{\textwidth}
\begin{minted}[bgcolor=lightyellow]{python}
#!/usr/bin/python
'''This version of the readserial program demonstrates
using python to write an output file'''

from datetime import datetime
import serial, io

outfile='/tmp/serial-temperature.tsv'

ser = serial.Serial(
   port='/dev/ttyUSB0',
   baudrate=9600,
)

sio = io.TextIOWrapper(
   io.BufferedRWPair(ser, ser, 1),
   encoding='ascii', newline='\r'
)

with open(outfile,'a') as f: #appends to existing file
   while ser.isOpen():
      datastring = sio.readline()
      #\t is tab; \n is line separator
      f.write(datetime.utcnow().isoformat() + '\t' + datastring + '\n')
      f.flush() #included to force the system to write to disk

ser.close()
\end{minted}
\end{minipage}
\end{lrbox}
\newpage
\begin{solution}
\noindent\usebox\logfile
\par
(see \texttt{python/exercises/example\_code/ldfsp.py} in your \texttt{ncas-isc} checkout)
\end{solution}
\newpage
\printsolutions
\end{document}
