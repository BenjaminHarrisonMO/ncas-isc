\documentclass[a4paper,11pt]{article}
\usepackage[top=30pt,bottom=30pt,left=48pt,right=46pt]{geometry}
\usepackage{exsheets}
\usepackage{exsheets-listings}
\usepackage{minted}
\definecolor{lightyellow}{HTML}{FFFFE0}
\usepackage{fontspec}
%NCAS Green colour
\definecolor{NCASgreen}{HTML}{186F4D}
\usepackage[colorlinks=true,urlcolor=NCASgreen]{hyperref}
\setsansfont{Open Sans}
\setmonofont{Ubuntu Mono}
\setmainfont{Liberation Serif}
\usepackage{mathtools}
\usepackage{unicode-math}
\setmathfont{Asana Math}

\begin{document}
\section{Logging Data From Serial Ports Exercise}

\begin{question}
Import the \texttt{serial} module and 
open the serial port with the appropriate parameters.
\end{question}
\newsavebox\serbasic
\begin{lrbox}{\serbasic}
\begin{minipage}{\textwidth}
\begin{minted}[bgcolor=lightyellow]{python}
#!/usr/bin/env python
import serial

ser = serial.Serial(
   port='/dev/ttyUSB0',
   baudrate=9600,
   bytesize=serial.EIGHTBITS,
   parity=serial.PARITY_NONE,
   stopbits=serial.STOPBITS_ONE
)
\end{minted}
\end{minipage}
\end{lrbox}


\begin{solution}
\noindent\usebox\serbasic
\end{solution}
\begin{question}
Get a reading from the temperature probe.
\end{question}
\begin{solution}
\texttt{print(ser.read(size=8))}
\par
"8" here is specific to the Papouch thermometer device.
\end{solution}
\begin{question}
Add a date and time reading to your output, using sensible choices for 
format, timezone, etc.
\end{question}
\begin{question}
Add a loop to your code to continuously log the reading and time. What would be a good exit condition? Hint: try \texttt{dir(serial.Serial)} to see what methods 
might be of use.
\end{question}
\newsavebox\serloop
\begin{lrbox}{\serloop}
\begin{minipage}{\textwidth}
\begin{minted}[bgcolor=lightyellow]{python}
while ser.isOpen():
   datastring = ser.read(size=8)
   print datetime.utcnow().isoformat(), datastring
\end{minted}
\end{minipage}
\end{lrbox}

\begin{solution}
Several ways, but the simplest is:

\noindent\usebox\serloop

\end{solution}

\begin{question}
Rewrite your code to use \texttt{readline()}.
\end{question}
\newsavebox\readlinebox
\begin{lrbox}{\readlinebox}
\begin{minipage}{\textwidth}
\begin{minted}[bgcolor=lightyellow]{python}
import io
…
sio = io.TextIOWrapper(io.BufferedRWPair(ser, ser, 1), encoding='ascii', newline='\r')
sio._CHUNK_SIZE =1

while ser.isOpen():
   datastring = sio.readline()
   print datetime.utcnow().isoformat(), datastring
\end{minted}
\end{minipage}
\end{lrbox}

\begin{solution}
\noindent\usebox\readlinebox
\end{solution}
\begin{question}
Alter your code to write the data out to a file.
\end{question}

\newsavebox\logfile
\begin{lrbox}{\logfile}
\begin{minipage}{\textwidth}
\inputminted[bgcolor=lightyellow]{python}{../../../python/exercises/example_code/ldfsp.py}
\end{minipage}
\end{lrbox}
\newpage
\begin{solution}
\noindent\usebox\logfile
\par
(see \texttt{python/exercises/example\_code/ldfsp.py} in your \texttt{ncas-isc} checkout)
\end{solution}
\newpage
\printsolutions
\newpage
\section{Writing and Plotting NetCDF files Exercise}
\begin{question}
Write a function to convert the time as written in
your datafile and return a Python \texttt{datetime}
object.
\end{question}p
\newsavebox\timeconv
\begin{lrbox}{\timeconv}
\begin{minipage}{\textwidth}
\begin{minted}[bgcolor=lightyellow]{python}
def convert_time(tm):
    tm = datetime.strptime(tm, "%Y-%m-%dT%H:%M:%S.%f")
    return tm
\end{minted}
\end{minipage}
\end{lrbox}
\begin{solution}
\noindent\usebox\timeconv

\texttt{strptime} is the opposite of \texttt{strftime} that we used earlier.
\end{solution}
\begin{question}
Write a function to convert the temperature as
written in your datafile and return a \texttt{float} in
Kelvin.

{\center $T_K = T_C + 273.15$}
\end{question}
\newsavebox\tempconv
\begin{lrbox}{\tempconv}
\begin{minipage}{\textwidth}
\begin{minted}[bgcolor=lightyellow]{python}
def convert_temp(temp):
    value = temp.strip("+").strip("C").lstrip("0")
    return float(value) + 273.15
\end{minted}
\end{minipage}
\end{lrbox}
\begin{solution}
\noindent\usebox\tempconv
\end{solution}
\begin{question}
Read your datafile into Python using the \texttt{csv} module such
that you end up with list object(s) containing floating-point
temperature in K and timestamps as Python \texttt{datetime} objects.


see: \url{https://docs.python.org/3/library/csv.html}
\end{question}
\newsavebox\csvread
\begin{lrbox}{\csvread}
\begin{minipage}{\textwidth}
\begin{minted}[bgcolor=lightyellow]{python}
infile='sample-serial-temperature-2h.tsv' #Or whatever your infile is called
outfile='sensor-data.nc'
from csv import reader

# Parse the data into python lists
times = []
temps = []

#open infile and read data into lists
with open(infile, 'rb') as tsvfile:
   tsvreader = reader(tsvfile, delimiter='\t')
   for row in tsvreader:
      times.append(convert_time(row[0]))
      temps.append(convert_temp(row[1]))
\end{minted}
\end{minipage}
\end{lrbox}
\begin{solution}
\noindent\usebox\csvread
\end{solution}
\begin{question}
\begin{itemize}
\item
  Create a \texttt{Dataset} (use the format \texttt{NETCDF4\_CLASSIC})
\item
  Convert your time series to a suitable CF-compliant series
\item
  Create a suitable \texttt{Dimension} for your time series
\item
  Create \texttt{Variable} objects for Temp and Time using appropriate
  units etc.
\item
  Assign appropriate metadata to the Temp \texttt{Variable} and and the
  \texttt{Dataset}
\item
  Add your time series and temp values to the \texttt{Dataset}
\item
  Close and write your \texttt{Dataset}. Test that it parses correctly
  with \texttt{ncdump}
\end{itemize}
\end{question}
\begin{solution}
See: \texttt{python/exercises/example\_code/write\_sensor\_data\_to\_netcdf.py} in your \texttt{ncas-isc} checkout.
\end{solution}

\begin{question}[name=Bonus Exercise]
You can do a quick-and-dirty plot with \texttt{ncview}:\par
\vspace{1em}
\texttt{ncview sensor\_data.nc}\par
\vspace{1em}
\noindent~This isn't publication quality. Produce a line plot of temperature vs time using \texttt{matplotlib}
and reading the data from your NetCDF file.
\end{question}
\begin{solution}
See: \texttt{python/exercises/example\_code/plot-netcdf.py}
\end{solution}
\newsavebox\cisexample
\begin{lrbox}{\cisexample}
\begin{minipage}{\textwidth}
\begin{verbatim}
cis plot temp:sensor_data.nc --xaxis time --yaxis temp \ 
    --title "Papouch Thermometer Data, 2017-02-22, UoL PRD" --xstep "0.010416"  \ 
    --output sensor_data_sample.svg
\end{verbatim}
\end{minipage}
\end{lrbox}
\begin{question}[name=Bonus Exercise]
\emph{``CIS is an open source command-line tool for easy collocation,
visualization, analysis, and comparison of diverse gridded and ungridded
datasets used in the atmospheric sciences''}
\vspace{1em}
It is based on python. Homepage:
\href{http://www.cistools.net/}{http://www.cistools.net/}
\vspace{1em}
\noindent\usebox\cisexample
\vspace{1em}
Experiment with CIS.
\end{question}
\begin{solution}
See: \texttt{python/presentations/logging-data-from-serial-ports/sensor\_data\_sample.svg}
\end{solution}

\newpage
\printsolutions[section]
\end{document}
